\section{Galois Theory for fields}

\begin{ex}
	Let $K \subset L$ be a Galois extension of fields, and let $I$ be a set of subfields $E \subset L$ with $K \subset E$ for which $[E:K] < \infty$ for every $E \in I$ and $\bigcup_{E \in I}  E = L$. Prove that $I$, when partially ordered by inclusion, is directed. 
\end{ex}

\begin{sol}
	Let $E',E \in I$. Then $EE'$ is a finite extension of $K$, because its generated by a the union of generators of $E$ and $E'$, which is a finite set of algebraic elements. Moreover, $EE'$ is separable ($L$ is Galois $\imp$ separable, and $EE' \subseteq L$ so all elements of $EE'$ are separable). Then, by the primitive element theorem, $EE' = K(\alpha)$ for a certain element $\alpha \in L$. Then, as $\bigcup_{E \in I} E = L$, $\exists F \in I$ such that $\alpha \in F$. Then, $EE' \subseteq F$ and so $E, E' \subseteq F$. This proves that $I$ is directed.
\end{sol}

\begin{ex}
	Let $K \subset L$ be a Galois extension of fields, and $I$ any directed set of subfields $E \subset L$ with $K \subset E$ Galois for which $\bigcup_{E \in I} E = L$. Prove that there is an isomorphism of profinite groups $\Gal(L/K) \cong \varprojlim_{E \in I} \Gal(E/K)$.
\end{ex}

\begin{sol}
	We will chech that the application $\phi: \Gal(L/K) \to \varprojlim_{E \in I} \Gal(E/K)$ defined by $\sigma \mapsto (\sigma|_E)_{E \in I}$ is the desired isomorphism of topological groups. First note that it is a well defined group morphism: As $K \subset E$ is Galois then $\sigma(E) = E$ for every $\sigma \in \Gal(L/K)$, so the restriction of $\sigma$ to $E$ is indeed an element of $\Gal(E/K)$. Moreover, if $F,E \in I$ with $F \subset E$, then $\phi(\sigma)_{E'} = \sigma|_{E'} = (\sigma|_E)|_{E'} = f_{EE'}(\phi(\sigma)_E)$.

	Now let's prove the continuity. For that we need to know the topology of $\varprojlim_{E \in I} \Gal(E/K)$. The topology will be induced by the product topology. A basis of open sets for the topology of $\prod_{E \in I} \Gal(E/K)$ is then 
	\[
		\left \{\prod_{E \notin J} \Gal(E/K) \times \prod_{E \in J} U_E \right \}
	\] 

	Where $J$ denotes a finite subset of $I$, and $U_E$ is an open set of $\Gal(E/K)$. Then, $U_E = \bigcup_{\sigma, F} U_{\sigma,F}$ for certain $\sigma \in \Gal(E/K)$, $K \subset F \subset E$, $[F:K] < \infty$. Therefore every basic open set of $\prod_{E \in I} \Gal(E/K)$ can be expressed as the union of sets $\prod_{E \notin J} \Gal(E/K) \times \prod_{E \in J} U_{\sigma^E, F^E}$, so those sets form a base of the topology of $\prod_{E \in I} \Gal(E/K)$. In conclusion, the following sets are a base of $\varprojlim_{E \in I} \Gal(E/K)$:
	\[
		\mathcal{B} = \left \{ \varprojlim_{E \in I} \Gal(E/K) \cap  \left ( \prod_{E \notin J} \Gal(E/K) \times \prod_{E \in J} U_{\sigma^E, F^E} \right ) \right \}
	\]

	Note that given $\sigma: E \to E$ we can extend it to $\overline{\sigma}: \overline{L} \to \overline{L}$ and then restrict it to $L$ to obtain $\sigma': L \to L$ such that $\sigma'|_E = \sigma$. Then, the antiimage by $\phi$ of an open set $U \in \mathcal{B}$ is the set $\tau \in \Gal(L/K) \text{ such that } \tau|_{F^E} = \sigma^E|_{F^E} \, \, \forall E \in J\} = \bigcap_{E \in J} U_{\sigma'^E, F^E}$, which is a finite intersection of open sets of $\Gal(L/K)$, and so it is open.

	Now let's prove injectivity of $\phi$: Let $\phi(\sigma) = \phi(\tau)$. It is enough to check that $\tau(\alpha) = \sigma(\alpha), \, \, \forall \alpha \in L$. Indeed, let $\alpha \in L$. Then as $\bigcup_{E \in I} E = L$, $\alpha \in E$ for a certain $E \in I$, and $\sigma|_E = \tau|_E$, which means $\sigma(\alpha) = \tau(\alpha)$ as desired.

	Let $(\sigma_E)_{E \in I} \in \varprojlim_{E \in I} \Gal(E/K)$. We will define $\sigma \in \Gal(L/K)$ as $\sigma(\alpha) = \sigma_E(\alpha)$ if $\alpha \in E$. It is clear that $\phi(\sigma) = (\sigma_E)_{E \in I}$, so we just have to check that $\sigma$ is well defined, that is, if $\alpha \in E, E'$, with $E, E' \in I$, then $\sigma_E(\alpha) = \sigma_E'(\alpha)$. But as the set is directed, $\exists F \in I$ such that $F \supset E, E'$ and so clearly $\alpha \in F$. As $(\sigma_E) \in \varprojlim_{E \in I} \Gal(E/K)$, then $\sigma|_E = \sigma_F|_E$ and $\sigma|_E' = \sigma_F|_E'$, so $\sigma_E(\alpha) = \sigma_F(\alpha) = \sigma_E'(\alpha)$ as desired.

	Finally note that $\Gal(L/K)$ is compact because it is profinite and $\varprojlim_{E \in I} \Gal(E/K)$ is Hausdorff, because each $\Gal(E/K)$ is Hausdorff and products and subspaces of Hausdorff are Hausdorff. Then $\phi$ is bijective and continuous group morphism, so it is an isomorphism of topological groups.

\end{sol}

\begin{ex}
	\begin{enumerate}[label=\alph*)]
		\item Let $K \subset L$ be a Galois extension of fields, with Galois group $G$. View $G$ as a subset of the set $L^L$ of all functions $L \to L$. Let $L$ be given the discrete topology and $L^L$ the product topology. Prove that the topology of the profinite group $G$ coincides with the relative topology inside $L^L$.

		\item Conversely, let $L$ be any field and $G \subset Aut(L)$ a subgroup that is compact when viewed as a subset of $L^L$ (topologized as in (a)). Prove that $L^G \subset L$ is Galois with Galois group $G$.

		\item Prove that any profinite group is isomorphic to the Galois group of a suitably chosen Galois extension of fields.
	\end{enumerate}
\end{ex}

\begin{sol}
	\begin{enumerate}[label=\alph*)]
		\item A basic open set of $L^L$ is of the form $U = \prod_{\alpha \in J} U_{\alpha} \times \prod_{\alpha \notin J, \alpha \in L} L$, where $J$ is a finite set of elements of $L$ and $U_{\alpha}$ is a subset of $L$ which is not the total. Then, a basic open set of $G$ (with topology induced by $L^L$) will be of the form $G \cap U = \{\sigma \in G \text{ such that } \sigma(\alpha) \in U_\alpha, \forall \alpha \in J \}$. As $J$ is a finite subset of $L$, then $K(J)$ is a finite extension. Let's note $F$ the normal closure of $K(J)$, which will also be a finite extension of $K$, which in addition is Galois. Let's consider the set $V = \{ \sigma \in \Gal(F/K) \text{ such that } \sigma(\alpha) \in U_{\alpha}, \forall \alpha \in J\}$. Then \[
			G \cap U = G \cap \left (V \times \prod_{L \supset E \neq F, E/K \text{ finite Galois }} \Gal(E/K) \right )
		\]

		And the right hand side is an open set of $G$ as a profinite group. This proves that the topology of $G$ as a profinite group is finer that that of $G$ as a subset of $L^L$.

		Reciprocally, let's take a basic open set of $G$ as a profinite group, $U = G \cap ( \prod_{E \in J} \Gal(E/K) \times \prod_{E \notin J} U_E)$. Each $E \in J$ can be expressed as $K(\alpha_E)$, by the primitive element theorem, so the action of $\sigma$ on $E$ is totally determined by the image of the element $\alpha_E$. Then, let $U_{\alpha_E} = \bigcup_{\sigma \in U_E} \sigma(\alpha)$, and the open set $U$ can now be described as 
		\[
			G \cap \left ( \prod_{E \in J} U_{\alpha_E} \times \prod_{\alpha \in L, \alpha \neq \alpha_E} L \right )
		\]

		which is an open set of $G$ as a subspace of $L^L$. This proves that the topology of $G$ as a subset of $L^L$ is finer than the topology of $G$ as a profinite group. In conclusion both topologies of $G$ are the same one.

		\item First let's prove that $L^G \subset L$ is algebraic: Indeed, let $\alpha \in L$ and let's take the cover of $G$ given by
		\[
			\left \{ \{\beta\} \times \prod_{\alpha' \neq \alpha} L \right \}_{\beta \in L}
		\] 

		As $G$ is compact as a subset of $L^L$ we can extract a finite subcovering from that covering, and that means that the orbit of $\alpha$ under $G$ is finite. Let $f(x) = \prod_{\beta \in G\alpha} (x-\beta)$. That polynomial is invariant under the action of $G$, so it has coefficients in $L^G$. Clearly $\alpha$ is a root of $f$, so $\alpha$ is algebraic. Then by definition $L^G \subset L$ is Galois, and we have that $G \subseteq \Gal(L/L^G)$, as every element of $G$ fixes $L^G$. As $G$ is compact and $L^L$ is Hausdorff, $G$ is closed in $L^L$, and then it is closed as a subgroup of $\Gal(L/L^G)$, by (a). Then, by the correspondence between closed subgroups of $\Gal(L/L^G)$ and field extensions, given by 2.3, we have that $\Gal(L/L^G) = G$.

		\item Let $K$ be any field and let $X$ be the set of conjugacy classes $\tau H$, where $H$ is an open normal subgroup of $G$. Let $G$ act on $X$ as follows: Given $\sigma \in G$, $\sigma(\tau H) = (\sigma \tau)H$. Let $L = K(X)$. Then the action of $G$ on $X$ induces an automorphism of $L$ for every element of $G$, so we have a natural map $G \to \mathrm{Aut}(L)$. This map is injective: Indeed, an element $\sigma \in G$ acts trivially on $X$ if and only if $\sigma \in H$, $\forall H$ open normal subgroup. As $G$ is profinite, then $G \cong \varprojlim G/H$, where $H$ runs through all the open normal subgroups of $G$ (this is a result in Cassels, Algebraic Number Theory, Chapter V, Corollary 1), and so the only $\sigma \in H \, \, \forall H$ is the identity. Then the only element acting trivially on $X$ is the identity, and so $G \to \mathrm{Aut}(L)$ is injective and we can view $G$ as a subset of $\mathrm{Aut}(L)$ (1).

		Note that every element of $L$ can be expressed as a quotient of polynomials with indeterminates as elements of $X$, so $\alpha$ has a finite orbit, because if $\mathcal{H}_{\alpha}$ is the set of groups with conjugacy classes apperaing in the expression of $\alpha$, then $|G\alpha| \leq \prod_{H \in \mathcal{H}_\alpha}[G:H]$. Then, given a basic open set of $L^L$, $U = \prod_{\alpha \in J} U_\alpha \times \prod_{\alpha \notin J} L$, with $J$ a finite subset of $L$. For $\alpha \in J$, $H \in \mathcal{H}$, consider all the conjugacy classes of $H$ that appear in the expression of a certain $\beta$, with $\beta \in U_\alpha$, and note it $U_H$. Then, the open set $U$ can be expressed as an open set of $G$ as a profinite group as follows:
		\[
			U = \prod_{\alpha \in J} \prod_{H \in \mathcal{H}_\alpha} U_H \times \prod_{H \notin \mathcal{H}_{\alpha}, \, \, \forall \alpha} G/H
		\]

		This shows that the topoogy of $G$ as a profinite group is finer that its topology as a subspace of $L^L$ (that is more general than the proof done in (a) of the same fact, because in (a) we knew that $G$ was a Galois group). Then, given an open cover of $G$ in $L^L$ it is also an open cover of $G$ as a profinite group, and as all profinite groups are compact, we can extract a finite covering. This proves that $G$ is compact when viewed as a subset of $L^L$. (2)

		Now $G$ satisfies the two conditions of (b), so $L/L^G$ is then a Galois with Galois group $G$.

	\end{enumerate}
\end{sol}


\begin{ex}

\end{ex}

\begin{ex}
	Let $K \subset L$ be a Galois extension of fields, $S \subset \Gal(L/K)$ any subset, and $E = \{x \in L : \forall \sigma \in S \sigma(x) = x\}$. Prove that $\Gal(L/E)$ is the closure of the subgroup of $\Gal(L/K)$ generated by $S$.
\end{ex}

\begin{sol}
	$\forall \sigma \in S, \, x \in E$, we have $\sigma(x) = x$, so $E$ is fixed by $S$ and so $<S> \subset \Gal(L/E)$. To chech $\overline{<S>} = \Gal(L/E)$, it is enough to check that $U_{\sigma,F} \cap <S> \neq \varnothing$ for every $\sigma \in \Gal(L/E)$ and every $F$. We will proceed as in the proof of the main theorem: Given a finite extension $K \subset F$, let $M$ be a finite Galois extension such that $F \subset M$. Let's restrict $<S>$ to $M$ to obtain $H'$ a subgroup of $\Gal(M/K)$. We have that $M^{H'} = E \cap M$, as both sides of the equality are the elements of $M$ fixed by $S$. We have $\sigma|_{M^{H'}} = Id$, so $\sigma|_M \in \Gal(M/M^{H'}) = H' = <S>|_M$.  Then, $\exists \tau \in <S>$ such that $\tau|_M = \sigma_M$, and restricting further to $F$ we have finally $U_{\sigma, F} \cap <S> \neq \varnothing$.
\end{sol}

\begin{ex}
	Let $K \subset L$ be a Galois extension of fields, $S \subset \Gal(L/K)$ and $H' \subset H \subset \Gal(L/K)$ closed subgroups with index$[H:H'] < \infty$. Prove that $L^H \subset L^{H'}$ is finite, and that $[L^{H'} : L^{H}] = $ index$[H:H']$. Which part of the conclusion is still true if $H', H$ are not necessarily closed?
\end{ex}

\begin{sol}
	$L^H \subset L^{H'}$ is a Galois extension. Indeed, it is algebraic, because $L$ is algebraic over $K$ so every element of $L^{H'}$ is algebraic over $K$ and therefore also over $L^{H}$. Moreover, $H$ is a subgroup of $\mathrm{Aut}(L^{H'})$ so $L^H \subset L^{H'}$ is Galois.

	As $H'$ is closed in $\Gal(L/K)$, it is also closed in $H$, and $H = \Gal(L/L^H)$. As $H'$ corresponds to a closed subgroup of finite index of $H$ (by hypothesis), then it is an open subgroup of $H$. Now, using the fundamental theorem 2.3(a), we have that $L^H \subset L^{H'}$ is finite, and $[L^{H'} : L^H] = \text{index}[H:H']$.

	If $H$, $H'$ are not necessarily closed, then every coset of $H'$ induces a morphism $\tau: L^{H'} \to L$ such that $L^H$ is invariant (that is, an $L^H$-immersion). The number of such immersions is the separablility index $[L^{H'}:L^{H}]$, which we knoe that divides the degree of the extension for finite extensions. Then, it still holds that $[L^{H'}:L^H] \geq \text{index}[H:H']$.
\end{sol}

\begin{ex}
	Let $K, L, F$ be subfields of a field $\Omega$, and suppose that $K \subset L$ is Galois and that $K \subset F$. Prove that $F \subset L · F$ is Galois, and that $\Gal(L · F/F) \cong  Gal(L/L \ F)$ (as topological groups).
\end{ex}

\begin{sol}
	Every element of $L$ is normal algebraic and separable. As these properties are conserved by adjuntion, then $LF = F(L)$ is also algebraic, separable and normal, so it is Galois. Now consider the application $\Phi: \Gal(LF/F) \to \Gal(L/L\cap F)$ defined by restriction $\Phi(\sigma) = \sigma|_L$.

	The application is well defined, as $L$ is normal, so $\sigma(L) = L$ and $\Phi(\sigma)(\alpha) = \sigma(\alpha) = \alpha$ for every element $\alpha \in L \cap F$. This proves that $\Phi(\sigma)$ is a indeed an element of $\Gal(L/L\cap F)$. Now let's check that we have an isomorphism of topological groups.

	First we prove injectivity: Let $\sigma \in \Gal(LF/F)$ such that $\sigma|_L = Id$. Then, $\forall \alpha \in L, \sigma(\alpha) = \alpha$. As $LF = F(L)$, then $\sigma$ is completely determined by its image over the elements of $L$, so if $\sigma|_L = Id$, then $\sigma = Id$. A similar argument works to prove surjectivity: Given $\tau \in \Gal(L/L \cap F)$ we define $\sigma \in \Gal(LF/F)$ as $\sigma(\alpha) = \sigma(\sum_{i = 1}^n a_i \alpha_i) = \sum_{i = 1}^n a_i \sigma(\alpha_i)$, and we clearly have $\Phi(\sigma) = \tau$.

	Now let's prove continuity. Let $U_{\sigma, E}$ be an open set of $\Gal(L/L \cap F)$. $U_{\sigma, E} = \{\tau \in \Gal(L/L \cap F) \text{ such that } \tau|_E = \sigma|_E\}$, with $[E:L \cap F] < \infty$. We have $\Phi^{-1}(U_{\sigma, E}) = \{ \tau \in \Gal(LF/F) \text{ such that } \tau|_E = \sigma|_E \}$. The image of $\tau$ on $E$ is completely determined by the image of a certain element $\alpha$, by the primitive element theorem. So let $E' = F(\alpha)$ and we have that $[E':F] < \infty$ and $\tau|_E' = \sigma'|_E'$, where $\sigma'$ is the only element of $\Gal(LF/F)$ such that $\Phi(\sigma') = \sigma$. This proves that $\Phi^{-1}(U_{\sigma, E}) = U_{\sigma', E'}$ which is open, and this proves the continuity. So we have a continuous bijective map between profinite groups, and therefore it is an isomorphism of topological groups.
	
\end{sol}

\begin{ex}
	
\end{ex}


\begin{ex}
	Let $K$ be a field. Prove that for every Galois extension $K \subset L$ the group $\Gal(L/K)$ is isomorphic to a quotient of the absolute Galois group of K.
\end{ex}

\begin{sol}
	Let $K \subset L$ be a Galois extension, and consider $\overline{L} = \overline{K}$ the algebraic closure. Then we have $K \subset L \subset K_s$ Galois extensions, and by the fundamental theorem (2.3 (d)) $\Gal(K_s/L)$ is a normal subgroup of $\Gal(K_s/K)$ and $\Gal(L/K) \cong \Gal(K_s/K)/\Gal(K_s/L)$.
\end{sol}

\begin{ex}
	A Steinitz number or supernatural number is a formal expression $a = \prod_{p \text{ prime}} p^{a(p)}$, where $a(p) \in \{0,1,2, \dots , \infty\}$ for each number $p$. If $a = \prod_{p}p^{a(p)}$ is a Steinitz number, we denote by $a\hat{\Z}$ the subgroup of $\hat{\Z}$ corresponding to $\prod_p p^{a(p)}\Z_p$ (with $p^{\infty}\Z_p = \{0\}$) under the isomorphism $\hat{\Z} \cong \prod_p \Z_p$.
	\begin{enumerate}[label=\alph*)]
		\item Prove that the map $a \mapsto a\hat{\Z}$ from the set of Steinitz numbers to the set of closed subgroups of $\hat{\Z}$ is bijective. Prove also that $a\hat{Z}$ is open if and only if $a$ is finite, i.e. $\sum_p a(p) < \infty$.

		\item Let $\mathbb{F}_q$ be a finite field, with algebraic closure $\overline{\mathbb{F}_q}$. For a Steinitz number $a$, let $\mathbb{F}_{q^a}$ be the set of all $x \in \mathbb{F}_q$ for which $[\mathbb{F}_q(x):\mathbb{F}_q]$ divides $a$ (in an obvious sense). Prove that the map $a \mapsto \mathbb{F}_{q^a}$ is a bijection from the set of Steinitz numbers to the set of intermediate fields of $\mathbb{F}_q \subset \overline{\mathbb{F}_q}$.
	\end{enumerate}
\end{ex}

\begin{sol}
	\item Injectivity of the application is clear. To show surjectivity, it is enough to prove that every closed subgroup of $\hat{\Z}$ is of the form $a\hat{\Z}$. Indeed, if $G$ is a closed subgroup of $\hat{\Z}$, then $G = \hat{\Z} \cap \prod{G_n}$, where $G_n$ is a subgroup of $\Z/n\Z$ for each $n$. When $n = p^m$, the only possibilities are $G_{p^m} = p^{k_m} \Z/p^m\Z$. Moreover, if $k_m \neq m$ for a certain $m$ we will have $G_{p^{m'}} = p^{k_m} \Z /p^{m'} \Z$. Let's define $a(p)$ as this value of $k_m$. Then, under the isomorphism $\hat{\Z} \cong \prod_p \Z_p$ the closed subgroup $G$ corresponds to $\prod_{p} p^{a(p)}\Z_p$, which is by definition $a \hat{\Z}$.

	It is clear that if $a$ is finite, then $a\hat{\Z}$ has finite index and so it is open. Reciprocally, using again problem 1.11, we know that $a\hat{\Z}$ is open if and only if $\exists n$ such that $\ker f_n: \hat{\Z} \to \Z/n\Z$ is a subgroup of $a\hat{\Z}$. Then, if $n = \prod p_i^{k_i}$, $\ker f_n = \prod_{p_i | n} p_i^{k_i} \Z_{p_i} \times \prod{p \nmid n} \Z_p$. $\ker f_n \subset a\hat{\Z} \iff$ $a(p) = 0$ for all $p \nmid n$ and $a(p_i) \leq k_i$ for all $p_i | n$. Then $\sum_p a(p) \leq \sum_{i = 1}^m k_i \leq \infty$, and so $a$ is finite.

	\item By (a) we have a correspondence between Steinitz numbers and the set of closed subgroups of $\hat{\Z}$. As $\Gal(\overline{\mathbb{F}_q}/ \mathbb{F}_q) \cong \hat{\Z}$, then theorem 2.3 gives a correspondence between Steinitz numbers and intermediate extensions $\mathbb{F}_q \subset E \subset \overline{\mathbb{F}_q}$ given by $a \mapsto \overline{\mathbb{F}_q}^{a\hat{\Z}}$. So we only need to check that $\overline{\mathbb{F}_q}^{a\hat{\Z}} = \mathbb{F}_{q^a}$.

	Let $x \in \mathbb{F}_{q^a}$. Then $[\mathbb{F}_q (x) : \mathbb{F}_q] = n$, and $n|a$ so $\Gal(\mathbb{F}_q(x)/\mathbb{F}_q) = \Z/n\Z$, and so $\Gal(\overline{\mathbb{F}_q}/\mathbb{F}_q(x)) = n \hat{\Z}$. Then as $n|a$ it is clear that $a \hat{\Z} \subset n\hat{\Z}$, and so $x \in \mathbb{F}_{q}^{a\hat{\Z}}$.

	Reciprocally, given $x \in \mathbb{F}_{q}^{a\hat{\Z}}$ let's consider the extension $\mathbb{F}_q \subset \mathbb{F}_q(x)$, which is finite and of degree a certain $n$. Then, $\mathbb{F}_q(x) \subset \mathbb{F}_q^{a\hat{\Z}}$, and so $\Gal(\overline{\mathbb{F}_q}/\mathbb{F}_q^{a\hat{\Z}}) \subset \Gal(\overline{\mathbb{F}_q}/\mathbb{F}_q(x))$, and $a\hat{\Z} \subset n\hat{\Z}$, which implies that $n | a$ and so $x \in \mathbb{F}_{q^a}$.
\end{sol}

\begin{ex}
	
\end{ex}






%19
\begin{ex}
	Let $K$ be a field, $K_s$ its separable closure, $m$ a positive integer not divisible by char($K$), and $\omega$ the number of $m$-th roots of unity in $K$.
	\begin{enumerate}[label=\alph*)]
		\item Let for $\tau \in \Gal(K_s/K)$ the integer $c(\tau)$ be such that $\tau(\varsigma_m) = \varsigma_m^{c(\tau)}$, where $\varsigma_m$ denotes a primitive $m$-th root of unity. Prove that $\omega$ is the greatest common divisor of $m$ and all numbers $c(\tau)-1$, $\tau \in \Gal(K_s/K)$.

		\item \textbf{Schienzel's Theorem}. Let $a \in K$. Prove that the splitting field of $X^m -a$ over $K$ is abelian over $K$ if and only if $a^{\omega} = b^m$ for some $b \in K$. [Hint for the only if part: if $\alpha^m = a$, prove that $\alpha^{c(\tau)}/\tau(\alpha) \in K^*$ for all $\tau$.] 
	\end{enumerate}
\end{ex}

%Ho he escrit però no n'estic segur del tot?????!!!
\begin{sol}
	\begin{enumerate}[label=\alph*)]
		\item Let $d$ be the least exponent such that $\varsigma_m^d \in K$. Then, the subgroup of the $m$-th roots of unity generated by $\varsigma_m^d$ has order $\omega$, and so $\omega d = m$. Then it is clear that $\omega | m$. Moreover, let $\tau \in \Gal(K_s/K)$ and as $\varsigma_m^d \in K$ we will have that $\tau(\varsigma_m^d) = \varsigma_m^{dc(\tau)} = \varsigma_m^d$. This means that $c(\tau)d \equiv d \pmod{m} \imp c(\tau)d \equiv d \pmod{d \omega} \imp c(\tau) \equiv 1 \pmod{\omega} \imp \omega | c(\tau)-1$. This proves that $\omega$ is a common divisor of $m$ and $c(\tau)-1$, for all $\tau \in \Gal(K_s/K)$. Now suppose that the greatest common divisor is not $\omega$, that is, that exists $k > 1$ such that $k\omega | m$ and $k\omega |c(\tau)-1$,  $\forall \tau$. Then let $k\omega d' = m$, where $d'k = d$. Note that $\forall \tau$ we will have $\tau(\varsigma_m^{d'}) = \varsigma_m^{c(\tau)d'}$. And as $c(\tau)-1 \equiv 0 \pmod{k\omega}$, then $d'c(\tau) \equiv d' \pmod{m}$ so $\varsigma_m^{d'} \in K$, which is a contradiction as $d' < d$. 

		\item Let $L$ be the splitting field of $X^m -a$. The roots of this polynomial are $\varsigma^k_m \alpha$, for a certain $\alpha$ such that $\alpha^m = a$. Then $L = K(\alpha, \varsigma_m)$, and every element $\tau \in \Gal(L/K)$ is totally determined by $\tau(\varsigma_m) = \varsigma_m^{c(\tau)}$ and $\tau(\alpha) = \varsigma_m^s \alpha$ for a certain $s$. Given an element $g \in \Gal(L/K)$ defined by $s,c(g)$, we have $g(\varsigma_m^k \alpha) =  g(\varsigma_m^k)g(\alpha) = \varsigma_m^{kc(g)}\varsigma_m^s \alpha$. Then every element of $\Gal(L/K)$ can be expressed as $g = \sigma \tau$, with $\sigma \in \Gal(L/K(\varsigma_m))$ and $\tau \in \Gal(L/K(\alpha))$. These subgroups are abelian: The first one is a cyclic group of order a divisor of $m$, and the second one is a sugbroup of the multiplicative group $\Z/m\Z$. Note that then $\Gal(L/K)$ is abelian if and only if arbitrary $\sigma, \tau$ belonging to these subgroups commute.

		Let $\sigma \in \Gal(L/K(\varsigma_m))$ such that $\sigma(\alpha) = \varsigma_m^s \alpha$ and $\tau \in \Gal(L/K(\alpha))$. Then, $\sigma \tau (\varsigma_m^k \alpha) = \varsigma_m^{kc(\tau) + s} \alpha$ and $\tau \sigma (\varsigma_m^k \alpha) = \varsigma_m^{c(\tau)(k + s)} \alpha$. Then the group is abelian if and only if $c(\tau)s \equiv s \pmod{m}$ for all the possible values of $s$ and $c(\tau)$.

		Now, if $a^{\omega} = b^m$, the cyclic group $\Gal(L/K(\varsigma_m))$ has order a divisor of $\omega$, as the order is the least divisor of $m$ such that $\alpha^k \in K$, and so $k | \omega$. Then, we have $\sigma^{\omega} = Id$ and so $s\omega \equiv 0 (mod m)$, which means that $d | s$. Then, we have $c(\tau) = 1 + a\omega$ and multiplying by $s = s'd$ we get $c(\tau)s = s + s'm$ so we have indeed that $c(\tau)s \equiv s \pmod{m}$ and so the group is abelian.

		Reciprocally, let's follow the indication of the hint. Consider $\frac{\alpha^{c(\tau)}}{\tau(\alpha)}$ and apply $\sigma \in \Gal(L/K)$ to this number. Let $\sigma(\alpha) = \varsigma_m^t \alpha$ and so we have that
		\[
			\sigma \left (\frac{\alpha^{c(\tau)}}{\tau(\alpha)} \right ) = \frac{\sigma(\alpha)^{c(\tau)}}{\tau(\sigma(\alpha))} = \frac{(\varsigma_m^t \alpha)^{c(\tau)}}{\varsigma_m^{tc(\tau)}\tau(\alpha)} = \frac{\alpha^{c(\tau)}}{\tau(\alpha)}
		\] 

		So it is invariant by action of $\Gal(L/K)$ and therefore it is an element of $K$. Now let's choose an element $\tau$ such that $\tau \in \Gal(L/K(\alpha))$ and so we will have $\alpha^{c(\tau)-1} \in K$. Obviously we have alse $\alpha^m \in K$ so this leads $\alpha^w \in K$.
	\end{enumerate}
\end{sol}

%20
\begin{ex}
	\begin{enumerate}[label=\alph*)]
		\item Prove that $Q \cap {M^{*}}^m = Q^{m/gcd(m,2)}$.

		\item Let $L_m = M(\alpha \in \overline{\Q} : \alpha^m \in \Q)$, for $m \in \Z_{>0}$. Prove that $M \subset L_m$ is Galois, and that there is an isomorphism of topological groups
		\[
			\Gal(L_m/M) \to \Hom(Q, E_m^{gcd(m,2)})
		\]

		mapping $\sigma$ to $\sigma(\alpha^{1/m})/\alpha^{1/m}$.

		\item Define $E_m \to E_n$ by $\varsigma \mapsto \varsigma^{m/n}$ for $n$ dividing $m$, and let $\hat{E} = \varprojlim E_n$ with respect to these maps. Prove that $\hat(E) \cong \hat{\Z}$ as topological groups.

		\item Prove that $M \subset L$ is Galois and that the isomorphisms in (b) combine to yield an isomorphism of topological groups
		\[
			\Gal(L/M) \to \Hom(Q, \hat{E}^2)
		\]

		here $\Hom(Q, \hat{E}^2)$ has the relative topology in $(\hat{E}^2)^Q$. Prove also that this Galois group is isomorphic to the product of a countably infinite collection of copies of $\hat{Z}$.
	\end{enumerate}
\end{ex}

%23
\begin{ex}
	\begin{enumerate}[label=\alph*)]
		\item Let $A$ be a local ring and $x \in A$ such that $x^2 = x$. Prove that $x = 0$ or $x = 1$.
		\item Prove that any ring isomorphism $\prod_{i = 1}^s A_i \to \prod-{j = 1}^t B_j$, where the $A_i$ and $B_j$ are local rings and $t,s < \infty$, is induced by a bijection $\sigma: \{1,2,\dots, s\} \to \{1,2,\dots, t\}$ and isomorphisms $A_i \to B_{\sigma(i)}$.
	\end{enumerate}
\end{ex}

\begin{sol}
	\begin{enumerate}[label=\alph*)]
		\item If $A$ is a local ring then $\forall x \in A$, either $x$ is a unit or $x$ is an element of the Jacobson radical. Then, if $x = x^2 \imp x(1-x) = 0$ we have two options: If $x$ is a unit, then $x^{-1}x(1-x) = 0 \imp (1-x) = 0 \imp x = 1$. If $x$ is not a unit then as an element of the Jacobson radical we have $1-xy$ is a unit $\forall y \in A$. In particular, taking $y = 1$ we have that $1-x$ is a unit, and so $x(1-x)(1-x)^{-1} = 0 \imp x = 0$.
		
		\item Let's denote $e_i \in \prod_{i = 1}^s A_i$ the element that has zeros at all positions except at position i, where it has a 1. Let $\phi: \prod_{i = 1}^s A_i \to \prod_{j = 1}^t B_j$ denote the isomorphism of the statement. Then, $\phi(e_i)^2 = \phi(e_i^2) = \phi(e_i)$, so we must have for each coordinate $j$ that $\phi(e_i)_j$ equals either 0 or 1, by part (a) of this problem (note that $B_j$ is a local ring.

		Moreover, by injectivity of the application $\phi$, it is impossible that all the coordinates equal $0$, because we would have that $\phi(e_i) = \phi(0) \imp e_i = 0$, which is a contradiction. Then, there is at least one coordinate such that $\phi(e_i)_j = 1$. Let's fix that $j$. By surjectivity of $\phi$, we have now that  
	\end{enumerate}
\end{sol}